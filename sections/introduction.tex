\section{Introduction}
\label{ch:Introduction}

% TODO: needs citations and rewriting (most of it is from GPT)

Forecasting is an integral part of modern data science, with applications ranging from energy management to urban planning. Among these, probabilistic forecasting offers a robust framework for understanding and quantifying uncertainty, making it particularly relevant for dynamic systems like (electrical) energy demand and urban bike traffic. Unlike traditional deterministic approaches, quantile forecasts provide point predictions and prediction intervals, which are vital for decision-making under uncertainty. For instance, a 95\% central prediction interval defined by the 2.5\% and 97.5\% quantiles provides reliable energy demand forecasts that can help optimize grid operations and reduce costs, while accurate bike count predictions improve city planning and public transportation efficiency.

This report generates probabilistic forecasts using traditional tabular regression models, specifically Gradient-Boosting Decision Tree (GBDT) algorithms like XGBoost \parencite{chen_xgboost_2016} and LightGBM \parencite{ke_lightgbm_2017}. These tree-based models are evaluated against simpler benchmark models, including a historical quantile-based approach and linear quantile regression. The aim is to systematically assess the strengths and limitations of different GBDT implementations for probabilistic forecasting tasks.

The evaluation framework employs time-series cross-validation to ensure robustness across various temporal contexts. The Pinball Loss, a standard metric for assessing probabilistic forecasts, serves as the primary evaluation criterion. However, in the context of probabilistic forecasting, it is important to assess model calibration in addition to the relative evaluation metrics. For this, a calibration plot provides visual insight, and empirical coverages can be used to assess prediction intervals.
In addition to numerical metrics, visual inspections of forecast distributions and analyses of feature importance provide qualitative insights into model behavior and explanatory power.

This study investigates these models in the context of two datasets: The hourly electrical energy consumption in Germany \parencite{noauthor_smard_nodate} and daily bike traffic in Karlsruhe. It seeks to draw actionable conclusions on the effectiveness of GBDT methods for real-world probabilistic forecasting challenges. The code is made available on Github.\footnote{\href{https://github.com/MoritzM00/proba-forecasting}{https://github.com/MoritzM00/proba-forecasting}}. All empirical results, including metrics and plots, can be viewed online using DVC Studio.\footnote{\href{https://studio.datachain.ai/user/MoritzM00/projects/proba-forecasting-jclqxio6ht}{https://studio.datachain.ai/user/MoritzM00/projects/proba-forecasting-jclqxio6ht}}

The report is organized into four sections as follows. First, \Cref{ch:Data} introduces the dataset, followed by a detailed description of how the problem is modeled in \Cref{ch:Methodology}. Third, the project results are discussed in \Cref{ch:Results}, and finally, the conclusions are drawn in \Cref{ch:Conclusion}.