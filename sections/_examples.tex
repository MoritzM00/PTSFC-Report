%% LaTeX2e class for seminar theses
%% sections/content.tex
%% 
%% Karlsruhe Institute of Technology
%% Institute for Program Structures and Data Organization
%% Chair for Software Design and Quality (SDQ)
%%
%% Dr.-Ing. Erik Burger
%% burger@kit.edu
\section{Introduction}
\label{ch:Introduction}

%% -------------------
%% | Example content |
%% -------------------

This is the SDQ seminar template.
For more information on the formatting of theses at SDQ, please refer to
\url{https://sdqweb.ipd.kit.edu/wiki/Ausarbeitungshinweise} or to your advisor.

\subsection{Example: Citation}
\label{sec:Introduction:Citation}
A citation: \cite{becker2008a} For referencing, see \autoref{sec:Introduction:Figures}

\subsection{Example: Figures}
\label{sec:Introduction:Figures}
\begin{figure}
\centering
\includegraphics[width=4cm]{images/sdqlogo}
\caption{SDQ logo}
\label{fig:sdqlogo}
\end{figure}

A reference: The SDQ logo is displayed in \autoref{fig:sdqlogo}. 
(Use \code{\textbackslash autoref\{\}} for easy referencing.) 

\subsection{Example: Tables}
\label{sec:Introduction:Tables}
\begin{table}
\centering
\begin{tabular}{r l}
\toprule
abc & def\\
ghi & jkl\\
\midrule
123 & 456\\
789 & 0AB\\
\bottomrule
\end{tabular}
\caption{A table}
\label{tab:atable}
\end{table}

\subsection{Example: Todo-Note}
Meaningless text.

\subsection{Example: Formula}
One of the nice things about the Linux Libertine font is that it comes with
a math mode package.
\begin{displaymath}
f(x)=\Omega(g(x))\ (x\rightarrow\infty)\;\Leftrightarrow\;
\limsup_{x \to \infty} \left|\frac{f(x)}{g(x)}\right|> 0
\end{displaymath}

%% --------------------
%% | /Example content |
%% --------------------

\section{First Content Section}
\label{ch:FirstContentSection}

%% -------------------
%% | Example content |
%% -------------------
The content chapters of your thesis should of course be renamed. How many
chapters you need to write depends on your thesis and cannot be said in general.

Check out the examples theses in the SDQWiki:

\url{https://sdqweb.ipd.kit.edu/wiki/Form_der_Ausarbeitung_bei_Seminaren}

Of course, you can split this .tex file into several files if you prefer. 


\subsection{First Subsection}
\label{sec:FirstContentSection:FirstSubSection}

\dots

\subsection{A Subsection}
\label{sec:FirstContentSection:FirstSubSubSection}

\dots


\section{Second Content Section}
\label{ch:SecondContentSection}

\dots

\subsection{First Subsection}
\label{sec:SecondContentSection:FirstSubsection}

\dots

\subsection{Second Subsection}
\label{sec:SecondContentSection:SecondSubsection}

\dots

Add additional content sections if required by adding new .tex files in the
\code{sections/} directory and adding an appropriate 
\code{\textbackslash input} statement in \code{thesis.tex}. 
%% ---------------------
%% | / Example content |
%% ---------------------