\newpage
\section{Models}
\label{ch:Models}

This chapter briefly describes the models that were used throughout the project. A standard practice in data science projects is to report metrics for simple baseline models to be able to compare more sophisticated approaches against them. The baseline models used here are described in \cref{sec:Models:Baselines}. As stated in the beginning, a focus is set on Gradient-Boosting decision trees, discussed in \cref{sec:Models:GBDT}.

All models discussed here optimize one quantile level at a time. Thus, when multiple quantile levels are of interest, several models must be trained. Because they do not optimize them jointly and strict monotonicity is not enforced between models, quantile crossing can occur. A simple solution is quantile sorting, in which the predictive quantiles are sorted so that no quantile crossing occurs.

\subsection{Baseline Models}
\label{sec:Models:Baselines}

The first baseline model is a historical quantile-based approach, referred to as \textit{Benchmark}. The Benchmark computes quantiles based on the last $n = 100$ observations that share the same weekday and hour, which is a straightforward method to incorporate simple seasonalities in the data. The second baseline model is a linear quantile regression \parencite{koenker_regression_1978} which means that the conditional $\tau$-th quantile of $Y$ given some random variables $X = (X_1, \ldots, X_p)^T$ with $p \in \mathbb{N}$ is modeled as
\begin{equation}
    q_\tau(Y | X) = \beta_1 X_1 + \cdots + \beta_p X_p \, .
\end{equation}
The coefficients $\beta_1, \ldots, \beta_p$ are determined by minimizing the Pinball Loss in \cref{eq:PinballLoss}.

\subsection{Gradient-Boosting Decision Trees}
\label{sec:Models:GBDT}

In Boosting, the general idea is to sequentially train an ensemble of base learners, where each base learner tries to improve upon the predictions of the previous one. \parencite[337]{hastie_elements_2009}. The base learner is often chosen to be a Decision Tree because it allows for very high flexibility and can cover a wide range of problems \parencite[350--352]{hastie_elements_2009}, thus popularizing Boosting Trees. Specifically, the final model is a sum of trees $T(x; \Theta_m)$ parametrized by $\theta_m$:
\begin{equation}
    f_M(x) = \sum_{i = 1}^M T(x; \Theta_m) \, ,
\end{equation}
where $M$ denotes the total number of steps or \enquote{boosts} taken. The parameters $\Theta_m$ at step $m$ are obtained by optimization of any differentiable loss function $\mathcal{L}$ in a forward stagewise manner \parencite[356--357]{hastie_elements_2009}:
\begin{equation}
    \label{eq:GBDT-optimization}
    \hat{\Theta}_m=\arg \min _{\Theta_m} \sum_{i=1}^N \mathcal{L}\left(y_i, f_{m-1}(x_i)+T(x_i ; \Theta_m)\right) \, ,
\end{equation}
where $f_{m-1}$ is current model.
In GBDT, this numerical optimization is done via Gradient descent. Hence, the individual tree predictions $T(x; \theta_m)$ can be seen as the direction of steepest descent, which is the negative gradient \parencite[359]{hastie_elements_2009}. For a detailed mathematical formulation of the numerical optimization procedure, see, for example, \cites[1189--1193]{friedman_greedy_2001}[353--360]{hastie_elements_2009}.

In the following subsections, essential characteristics of each of the three popular GBDT-Libraries are discussed. These include LightGBM \parencite{ke_lightgbm_2017}, XGBoost \parencite{chen_xgboost_2016}  and CatBoost \parencite{prokhorenkova_catboost_2017}.

\subsubsection{XGBoost}

In \textsc{eXtreme Gradient Boosting} (XGBoost), \cref{eq:GBDT-optimization} is modified to include a regularization term that controls the complexity of the trees \parencite[2]{chen_xgboost_2016}. This effectively leads to a pre-pruning strategy because it reduces the scoring function used for evaluating split candidates \parencites[1941--1942]{bentejac_comparative_2021}.\footnote{The hyperparameter that controls this regularization effect is called $\gamma$ in the original paper. It corresponds to the minimum gain needed to split a node.}
Furthermore, XGBoost proposes a new approximate split finding algorithm to avoid iterating over all possible splits, which is infeasible if the dataset does not fit into memory \parencite[Chapter~3]{chen_xgboost_2016}. Instead, the basic idea is to divide a continuous feature into buckets based on percentiles of the feature distribution and then to find the best split candidate based on aggregated statistics \parencite[Chapter~3.2]{chen_xgboost_2016}.
As XGBoost was designed to work efficiently with larger-than-memory datasets, it includes many technical optimizations, such as the compressed column format for parallel sorting and cache-aware access \parencite[Chapter~4]{chen_xgboost_2016}.

\subsubsection{LightGBM}

\textsc{Light Gradient-Boosting Machine} (LightGBM) proposes two techniques -- \textit{Gradient-Based One-Side Sampling} (GOSS) and \textit{Exclusive Feature Bundling} (EFB) -- to improve upon the original Gradient-Boosting algorithm \parencite[1]{ke_lightgbm_2017}. EFB only applies to categorical features and is thus omitted here for brevity. The main motivation of GOSS is that some training instances are more important to further reduce the loss than other instances. The importance can be measured by the magnitude of the gradient, because large gradients indicate that the prediction of the model should change more dramatically. Thus, the model already predicts instances with small gradients well. Hence, in GOSS, gradients are sorted descendingly to then divide instances into the top $a$ \% (e.g. $a=20$) and bottom $100 - a$ \% based on their gradient. Random subsampling is done only on the bottom side, hence the name \textit{one-side sampling}.\footnote{However, in the LightGBM package, GOSS is disabled by default but can be activated by setting \texttt{boosting\_type=goss} in the sklearn-API or by setting \texttt{data\_sample\_strategy=goss} in the LightGBM-native API.}

Compared with other GBDT implementations, LightGBM consistently trains the fastest \parencite[1951--1954]{bentejac_comparative_2021} -- probably due to a highly-optimized histogram-based split finding algorithm paired with leaf-wise growth of decision trees. Leaf-wise growth (also known as best-first) means that the node that maximally reduces the impurity (e.g. information gain) is splitted \parencite[3]{shi_best-first_2007}. In contrast, the standard method to grow decision trees, which is used by XGBoost as well, is depthwise (or: levelwise), meaning that all nodes in one level are expanded, e.g., left-to-right, before the tree can grow deeper. 

\subsubsection{CatBoost}

\textsc{Categorical Boosting} (CatBoost) distinguishes itself from other GBDT implementations by a variation called \textit{ordered boosting}, which is supposed to prevent target leakage and the prediction shift caused by it. In traditional GBDT, instances used to estimate and then subsequently minimize the gradient are identical, thus introducing target leakage. The authors propose to calculate residuals for instances with models that were not trained on these instances using a permutation technique \parencites[5]{prokhorenkova_catboost_2017}[1943--1944]{bentejac_comparative_2021}. In addition, they propose a new technique for handling categorical features, making it especially viable when there are many categorical features in the data set \parencite[1]{prokhorenkova_catboost_2017}. Typically, CatBoost with ordered boosting trains the slowest among the three implementations discussed here, but some authors found that CatBoost works well in most cases without extensive hyperparameter tuning \parencite[26]{florek_benchmarking_2023}. Thus, the higher training might be acceptable when factoring in the lower amount of time spent on hyperparameter search. Lastly, CatBoost differs in tree construction as it builds symmetric trees by using the same split condition on each level. On the other hand, trees built by LightGBM and XGboost can be asymmetric in general \parencite[8]{florek_benchmarking_2023}.

