\section{Configuration for GBDT models}

In this section, the configuration for the GBDT models is explained in more detail. \Cref{tab:model-config} shows the configuration for the three models on the Bike Target. For these models, the following parameters are used to control the behavior and complexity in the gradient-boosting algorithm described in \cref{sec:Models:GBDT}: (NEED CITATION)

\begin{description}
  \item[\textbf{n\_estimators}] The number of trees (boosting rounds) built during training. Increasing this number can improve model performance but may also increase the risk of overfitting and computational cost.
  
  \item[\textbf{learning\_rate}] A shrinkage factor that scales the contribution of each tree. Lower values typically require more trees for convergence but can lead to better generalization by preventing overly aggressive updates.
  
  \item[\textbf{max\_depth}] The maximum depth of each individual tree. Deeper trees allow the model to capture more complex interactions but can also lead to overfitting. This parameter helps in controlling the model’s complexity. Note that the effectiveness of this parameter depends on how the trees are constructed (e.g. leaf-wise vs. depth-wise).

  \item[\textbf{num\_leaves}] The maximum number of leaves (i.e. terminal nodes) per tree. A larger value increases the model’s capacity but may also lead to overfitting. This is the primary and preferred way of limiting tree complexity in LightGBM due to its leaf-wise (best-first) node expansion.
  
  \item[\textbf{tree\_method}] (XGBoost-specific) The algorithm used for constructing trees. For example, the \texttt{hist} method builds histograms to accelerate training and reduce memory usage. Other available algorithms are \texttt{exact} (full enumeration of all split points) and \texttt{approx} (Approximate algorithm as outlined in \cite[Chapter~3.2]{chen_xgboost_2016}).
  
  \item[\textbf{grow\_policy}] The strategy for expanding trees during training. Options such as \texttt{depthwise} (used in XGBoost) or \texttt{SymmetricTree} (used in CatBoost) influence the structure of the trees and can affect performance. XGboost also supports leaf-wise growth. CatBoost supports all three options. This parameter is not available in LightGBM as it only supports leaf-wise growth.
  
  \item[\textbf{colsample\_bylevel}] The fraction of features randomly sampled for each level of the tree. This parameter introduces randomness into the model by only evaluating a subset of features for splitting, which can help reduce overfitting by de-correlating the trees.
  
  \item[\textbf{colsample\_bytree}] Similar to \texttt{colsample\_bylevel}, this parameter can also help prevent overfitting by only using a fraction of features for split evaluation. For this parameter, the subsampling is done once in each boosting iteration.

  \item[\textbf{min\_child\_samples}] The minimum number of data samples required in a leaf. This parameter acts as a regularizer by ensuring that leaves have enough data to make reliable predictions.
  
  \item[\textbf{reg\_lambda}] The L2 regularization term on leaf weights. Higher values impose a stronger penalty on large weights, which can help control overfitting.
  
\end{description}
\label{appendix:GBDT-configs}
\begin{table}[htbp]
  \centering
  \begin{tabular}{@{} lccc @{}}
    \toprule
    \textbf{Parameter} & \textbf{XGBoost} & \textbf{LightGBM} & \textbf{CatBoost} \\ 
    \midrule
    \multicolumn{4}{l}{\textbf{Feature-specific parameters}} \\
    Lagged target values& $1,2,6,7,14,21,28$ & $1,2,6,7,8,14,21,28$ & $1,2,6,7,14,21,28$ \\
    Include seasonal encodings & true & true & true \\
    Use cyclical Encodings & true & true & true \\
    Include rolling stats & false & false & false \\
    Lagged exogenous variables & All & All & All \\
    \midrule
    \multicolumn{4}{l}{\textbf{GBDT-specific parameters}} \\
    n\_estimators & 250 & 250 & 250 \\
    num\_leaves & -- & 15 & -- \\
    min\_child\_samples & -- & 50 & -- \\
    grow\_policy & depthwise & leafwise & SymmetricTree \\
    learning\_rate & 0.1 & 0.1 & 0.1 \\
    max\_depth & 6 & 6 & 6 \\
    colsample\_bylevel & 0.6 & -- & 0.6 \\
    colsample\_bytree & -- & 0.6 & -- \\
    subsample & 0.9 & 0.9 & 0.9 \\
    reg\_lambda & -- & 1 & 1e-2 \\
    \bottomrule
  \end{tabular}
  \caption{Configuration for the three GBDT implementations XGBoost, LightGBM, and CatBoost.}
  \label{tab:model-config}
\end{table}

\newpage
\section{Additional Visualizations}

\begin{figure}[htbp]
    \centering
    \begin{subfigure}[b]{0.95\textwidth}
        \centering
        \includesvg[width=1\linewidth]{images/catboost-energy-plot-1.svg}
        \caption{Forecasts for the week starting on January 2nd, 2024. At this time of the year, a higher uncertainty is expected due to holiday effects. This is also reflected in the slightly wider 50\% and 95\% prediction intervals. However, it is still at times too narrow and completely misses the actual time-series shown in black.}
        \label{fig:forecast-plot-1}
    \end{subfigure}
    \begin{subfigure}[b]{0.95\textwidth}
        \centering
        \includesvg[width=1\linewidth]{images/catboost-energy-plot-2.svg}
        \caption{Forecasts for a regular week in September 2024. The prediction intervals are slightly narrower than for the week shown in (a). However, they overestimate all quantiles for many timesteps in this week. The problem is exaggerated for the weekend starting with September 21, 2024, where the model overestimates the quantiles for almost all timesteps.}
        \label{fig:forecast-plot-2}
    \end{subfigure}
    \caption{Two forecast visualizations of the best model on the Energy Target, made during the TSCV for model selection as described in \cref{sec:Methodology:Evaluation}. It shows two TSCV folds, corresponding to two weeks in this setup. In black, the actual time-series is shown, and the dotted blue line is the predictive median. The shaded area shows the 50\% and 95\% prediction interval. (a) The first week of 2024 starting on Tuesday, 2nd, and (b) a regular week without holiday effects. It shows that the prediction intervals are often too narrow, which means the model is overconfident with its predictions. This could be due to the inclusion of too many lags, which decreases coverage levels. However, the model learned the higher uncertainty due to the holiday and seasonality effects in the first week of a year reflected by wider prediction intervals.}
    \label{fig:Forecast-visualization}
\end{figure}


\begin{figure}[htbp]
    \centering
    % First row: two images
    \begin{subfigure}[b]{0.5\textwidth}
        \centering
        \includesvg[width=\linewidth]{images/bikes_forecast/small-orfe_output_eval_plots_bikes_forecast_3.svg}
        \caption{Linear Quantile Regression}
        \label{fig:small-orfe}
    \end{subfigure}%
    \begin{subfigure}[b]{0.5\textwidth}
        \centering
        \includesvg[width=\linewidth]{images/bikes_forecast/armed-play_output_eval_plots_bikes_forecast_3.svg}
        \caption{Benchmark}
        \label{fig:armed-play}
    \end{subfigure}

    % second row 
    \begin{subfigure}[b]{0.5\textwidth}
        \centering
        \includesvg[width=\linewidth]{images/bikes_forecast/mangy-flux_output_eval_plots_bikes_forecast_3.svg}
        \caption{LightGBM}
        \label{fig:mangy-flux}
    \end{subfigure}%
    \begin{subfigure}[b]{0.5\textwidth}
        \centering
        \includesvg[width=\linewidth]{images/bikes_forecast/couth-ruby_output_eval_plots_bikes_forecast_3.svg}
        \caption{XGBoost}
        \label{fig:couth-ruby}
    \end{subfigure}
    % third row
    \begin{subfigure}[b]{0.5\textwidth}
        \centering
        \includesvg[width=\linewidth]{images/bikes_forecast/civil-leas_output_eval_plots_bikes_forecast_3.svg}
        \caption{Ordered CatBoost}
        \label{fig:civil-leas}
    \end{subfigure}%
    \begin{subfigure}[b]{0.5\textwidth}
        \centering
        \includesvg[width=\linewidth]{images/bikes_forecast/blear-dita_output_eval_plots_bikes_forecast_3.svg}
        \caption{Plain CatBoost}
        \label{fig:blear-dita}
    \end{subfigure}%
    
    \caption{Forecast visualizations of six different models during TSCV on the Bikes Target, showcasing some characteristics of each model.}
    \label{fig:bikes_forecast}
\end{figure}

\begin{figure}[htbp]
    \centering
    % First row: two images
    \begin{subfigure}[b]{0.5\textwidth}
        \centering
        \includesvg[width=\linewidth]{images/energy_forecast/small-orfe_output_eval_plots_energy_forecast_3.svg}
        \caption{Linear Quantile Regression}
        \label{fig:small-orfe-energy}
    \end{subfigure}%
    \begin{subfigure}[b]{0.5\textwidth}
        \centering
        \includesvg[width=\linewidth]{images/energy_forecast/armed-play_output_eval_plots_energy_forecast_3.svg}
        \caption{Benchmark}
        \label{fig:armed-play-energy}
    \end{subfigure}

    % Second row: two images
    \begin{subfigure}[b]{0.5\textwidth}
        \centering
        \includesvg[width=\linewidth]{images/energy_forecast/mangy-flux_output_eval_plots_energy_forecast_3.svg}
        \caption{LightGBM}
        \label{fig:mangy-flux-energy}
    \end{subfigure}%
    \begin{subfigure}[b]{0.5\textwidth}
        \centering
        \includesvg[width=\linewidth]{images/energy_forecast/couth-ruby_output_eval_plots_energy_forecast_3.svg}
        \caption{XGBoost}
        \label{fig:couth-ruby-energy}
    \end{subfigure}
    
    % Third row: two images
    \begin{subfigure}[b]{0.5\textwidth}
        \centering
        \includesvg[width=\linewidth]{images/energy_forecast/civil-leas_output_eval_plots_energy_forecast_3.svg}
        \caption{Ordered CatBoost}
        \label{fig:civil-leas-energy}
    \end{subfigure}%
    \begin{subfigure}[b]{0.5\textwidth}
        \centering
        \includesvg[width=\linewidth]{images/energy_forecast/blear-dita_output_eval_plots_energy_forecast_3.svg}
        \caption{Plain CatBoost}
        \label{fig:blear-dita-energy}
    \end{subfigure}%
    
    \caption{Forecast visualizations of six different models during TSCV on the Energy Target, showcasing some characteristics of each model.}
    \label{fig:energy_forecast}
\end{figure}

\begin{figure}[htbp]
    \centering
    % First row: two calibration curves
    \begin{subfigure}[b]{0.5\textwidth}
        \centering
        \includesvg[width=\linewidth]{images/bikes_calibration/small-orfe_output_eval_plots_bikes_calibration_curve.svg}
        \caption{Linear Quantile Regression Calibration}
        \label{fig:small-orfe-calibration}
    \end{subfigure}%
    \begin{subfigure}[b]{0.5\textwidth}
        \centering
        \includesvg[width=\linewidth]{images/bikes_calibration/armed-play_output_eval_plots_bikes_calibration_curve.svg}
        \caption{Benchmark Calibration}
        \label{fig:armed-play-calibration}
    \end{subfigure}
    
    % Second row: two calibration curves
    \begin{subfigure}[b]{0.5\textwidth}
        \centering
        \includesvg[width=\linewidth]{images/bikes_calibration/mangy-flux_output_eval_plots_bikes_calibration_curve.svg}
        \caption{LightGBM Calibration}
        \label{fig:mangy-flux-calibration}
    \end{subfigure}%
    \begin{subfigure}[b]{0.5\textwidth}
        \centering
        \includesvg[width=\linewidth]{images/bikes_calibration/couth-ruby_output_eval_plots_bikes_calibration_curve.svg}
        \caption{XGBoost Calibration}
        \label{fig:couth-ruby-calibration}
    \end{subfigure}
    
    % Third row: two calibration curves
    \begin{subfigure}[b]{0.5\textwidth}
        \centering
        \includesvg[width=\linewidth]{images/bikes_calibration/civil-leas_output_eval_plots_bikes_calibration_curve.svg}
        \caption{Ordered CatBoost Calibration}
        \label{fig:civil-leas-calibration}
    \end{subfigure}%
    \begin{subfigure}[b]{0.5\textwidth}
        \centering
        \includesvg[width=\linewidth]{images/bikes_calibration/blear-dita_output_eval_plots_bikes_calibration_curve.svg}
        \caption{Plain CatBoost Calibration}
        \label{fig:blear-dita-calibration}
    \end{subfigure}
    
    \caption{Calibration curves of six different models on the Bikes Target, illustrating the calibration performance of each model.}
    \label{fig:bikes_calibration}
\end{figure}

\begin{figure}[htbp]
    \centering
    % First row: two images
    \begin{subfigure}[b]{0.5\textwidth}
        \centering
        \includesvg[width=\linewidth]{images/energy_pinball/small-orfe_output_energy_pinball_losses.svg}
        \caption{Linear Quantile Regression Pinball Losses}
        \label{fig:small-orfe-pinball-energy}
    \end{subfigure}%
    \begin{subfigure}[b]{0.5\textwidth}
        \centering
        \includesvg[width=\linewidth]{images/energy_pinball/armed-play_output_energy_pinball_losses.svg}
        \caption{Benchmark Pinball Losses}
        \label{fig:armed-play-pinball-energy}
    \end{subfigure}
    
    % Second row: two images
    \begin{subfigure}[b]{0.5\textwidth}
        \centering
        \includesvg[width=\linewidth]{images/energy_pinball/mangy-flux_output_energy_pinball_losses.svg}
        \caption{LightGBM Pinball Losses}
        \label{fig:mangy-flux-pinball-energy}
    \end{subfigure}%
    \begin{subfigure}[b]{0.5\textwidth}
        \centering
        \includesvg[width=\linewidth]{images/energy_pinball/couth-ruby_output_energy_pinball_losses.svg}
        \caption{XGBoost Pinball Losses}
        \label{fig:couth-ruby-pinball-energy}
    \end{subfigure}
    
    % Third row: two images
    \begin{subfigure}[b]{0.5\textwidth}
        \centering
        \includesvg[width=\linewidth]{images/energy_pinball/civil-leas_output_energy_pinball_losses.svg}
        \caption{Ordered CatBoost Pinball Losses}
        \label{fig:civil-leas-pinball-energy}
    \end{subfigure}%
    \begin{subfigure}[b]{0.5\textwidth}
        \centering
        \includesvg[width=\linewidth]{images/energy_pinball/blear-dita_output_energy_pinball_losses.svg}
        \caption{Plain CatBoost Pinball Losses}
        \label{fig:blear-dita-pinball-energy}
    \end{subfigure}
    
    \caption{Pinball Losses for six different models on the Energy Forecast.}
    \label{fig:energy_pinball}
\end{figure}

\begin{figure}[htbp]
    \centering
    % First row: two images
    \begin{subfigure}[b]{0.5\textwidth}
        \centering
        \includesvg[width=\linewidth]{images/bikes_pinball/small-orfe_output_bikes_pinball_losses.svg}
        \caption{Linear Quantile Regression Pinball Losses}
        \label{fig:small-orfe-pinball-bikes}
    \end{subfigure}%
    \begin{subfigure}[b]{0.5\textwidth}
        \centering
        \includesvg[width=\linewidth]{images/bikes_pinball/armed-play_output_bikes_pinball_losses.svg}
        \caption{Benchmark Pinball Losses}
        \label{fig:armed-play-pinball-bikes}
    \end{subfigure}
    
    % Second row: two images
    \begin{subfigure}[b]{0.5\textwidth}
        \centering
        \includesvg[width=\linewidth]{images/bikes_pinball/mangy-flux_output_bikes_pinball_losses.svg}
        \caption{LightGBM Pinball Losses}
        \label{fig:mangy-flux-pinball-bikes}
    \end{subfigure}%
    \begin{subfigure}[b]{0.5\textwidth}
        \centering
        \includesvg[width=\linewidth]{images/bikes_pinball/couth-ruby_output_bikes_pinball_losses.svg}
        \caption{XGBoost Pinball Losses}
        \label{fig:couth-ruby-pinball-bikes}
    \end{subfigure}
    
    % Third row: two images
    \begin{subfigure}[b]{0.5\textwidth}
        \centering
        \includesvg[width=\linewidth]{images/bikes_pinball/civil-leas_output_bikes_pinball_losses.svg}
        \caption{Ordered CatBoost Pinball Losses}
        \label{fig:civil-leas-pinball-bikes}
    \end{subfigure}%
    \begin{subfigure}[b]{0.5\textwidth}
        \centering
        \includesvg[width=\linewidth]{images/bikes_pinball/blear-dita_output_bikes_pinball_losses.svg}
        \caption{Plain CatBoost Pinball Losses}
        \label{fig:blear-dita-pinball-bikes}
    \end{subfigure}
    
    \caption{Pinball Losses for six different models on the Bikes Forecast.}
    \label{fig:bikes_pinball}
\end{figure}